\documentclass{hausarbeit_philosophie}
\addbibresource{literatur.bib}

\title{Titel der Hausarbeit}
\semester{Sommer-/Wintersemester 20xy/20xy}
\name{Vorname Name}
\matrikelnummer{Matrikelnummer}
\hauptnebenfach{Hauptfach, Nebenfach}
\fachsemester{Fachsemester}
\email{email@uni-erfurt.de}
\veranstaltung{Titel der Veranstaldung}
\dozent{Titel und Namen der Dozentin/des Dozenten}
\modul{Modulbezeichnung}
\leistungart{Modulprüfung, qualifizierte Teilnahme}
\abgabe{\today}
\zeichenzahl{Anzahl der Zeichen ohne Leerzeichen}

\begin{document}

\maketitle
\tableofcontents
\newpage

\section{Überschrift Kategorie 1 (Überschrift I)}
Das ist ein Absatz ohne Einzug. In dieser Formatvorlage heißt die Formatierung "Standard ohne Einzug". Wählen Sie dieses Format, wenn Sie einen Absatz nach Überschriften oder nach Blockzitaten beginnen.

Das ist ein Absatz mit Einzug. Die Formatierung heißt "Standard mit Einzug". Diese Formatierung werden Sie wahrscheinlich am häufigsten gebrauchen. Der Einzug dient der Abgrenzung zum vorangegangenen Absatz und hilft, den Fließtext gut lesen zu können. Betrachten Sie Absätze als Sinnabschnitte: Sie leiten einen Gedanken ein, führen diesen aus und leiten zum nächsten Gedanken bzw. Sinnabschnitt über. Absätze können unterschiedlich lang sein. Sie bestehen jedoch nie aus einem einzigen Satz. Vermeiden Sie außerdem seitenlange Absätze.

\subsection{Das ist eine Überschrift der Kategorie 2 (Überschrift II)}
Kurze Zitate wie das ausgedachte von Max Mustermann, wonach Zitate "in den Fließtext gehören" \cite[14]{MusZit}, gehören in den Fließtext. Zitatnachweise indirekter Zitate werden mit 'vgl.' eingeleitet \vglcite[276]{MusTab}. Fußnoten sind für Literaturhinweise, erläuternde Anmerkungen oder weiterführende Diskussionen reserviert.\footnote{Das ist eine Fußnote.} Längere Zitate, die mindestens drei Zeilen umfassen, werden als Blockzitat eingefügt:

\blockzitat{Das ist ein Blockzitat mit gleichnamiger Bezeichnung in der Formatvorlage. Blockzitate sind direkte Zitate, die im Fließtext mehr als drei Zeilen [Hervorhebung M.M.] einnähmen. Blockzitate stehen nicht in Anführungszeichen. Der Nachweis erfolgt im Anschluss an das Zitat. Sie müssen in Ihrem Fließtext auf das Blockzitat eingehen, es problematisieren oder interpretieren. Das Blockzitat steht niemals für sich und ersetzt nie Ihren eigenen Beitrag. \cite[96]{MusBlock}}

Nach dem Blockzitat geht es mit einem Text ohne Einzug weiter. Kapitel enden nie mit einem Blockzitat.

Direktes Zitat mit Siglum: "Es ist überall nichts in der Welt, ja überhaupt auch außer derselben zu denken möglich, was ohne Einschränkung für gut könnte gehalten werden, als allein ein guter Wille." \cite[393]{GMS}.

Kant ist außerdem der Meinung, dass indirekte Zitate mit Siglum in diesem Satz demonstriert werden \vglcite[999]{GMS}.

\printbibliography

\end{document}
